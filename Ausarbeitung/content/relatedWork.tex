This chapter describes the history of hacking and password security in \prettyref{sec:historicOverview}. Hacking developed in the early 20$^{th}$ century and became more important over time. \prettyref{sec:related} introduces some work about security and safe encryption algorithms. 
The Oxford English Dictionary appeared around 1200 and meant to 'cut with heavy blows in an irregular or random fashion'. The word 'hack' was first used for fussing with machines in the M.I.T. in April 1955. 

\todo{Wortherkunft: http://www.newyorker.com/tech/elements/a-short-history-of-hack}

\todo{describe cracking: http://www.catb.org/~esr/jargon/html/C/cracker.html}

\section{Historical Overview}
\label{sec:historicOverview}

The following historical overview of computer security hacker history covers some important events in the development of hacking and cracking. It started in 1932 by the polish cryptologists Marian Rejewski, Henryk Zygalski and Jerzy Rózycki who broke the Enigma machine code. The Enigma machine is a electro-mechanical rotor cipher machine that was used in the second world war in order to encrypt the information exchange of the german military. Despite permanent improvement of the encryption algorithm, were the allies able to encrypt nearly every message. 
Alan Turing, Gordon Welchman and Harold Keen together developed the Turing-Welchman-Bombe based on the work of Rejewski et al. Since the Enigma uses a reliably small key space, it was possible to brute force the possibilities. 
\todo{cite - alles wikipedia...?}

Phreaking came up in the 1960s. This means to manipulate phone systems by using various audio frequencies. By using a specific frequency the possibility to make free calls has been discovered.

In the late 1970s, Kevin Mitnick breaks into the first major computer system. In 1980 the FBI examines a breach of security at National CSS, a time-sharing firm. Time-sharing means sharing computer resources by allowing many users to interact with a single computer on the same time. This idea made it possible to use a computer without owing it. The hacking scandal was not motivated by a malicious intent or damage. Nevertheless, it was possible to see which interesting data might be found. Afterwards, hackers were described as technical, skilled experts that are often young computer programmers. 



\section{Related Papers}
\label{sec:related}

\todo{work about security and safe encryption algorithms}