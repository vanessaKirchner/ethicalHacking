
Enjoying to explore details of a programmable system and how to stretch the capabilities is often described as hacking. Additionally, the understanding of the internal working of a system, computers or computer networks is a feature of hacking. \cite{jargon}

This definition can be interpreted in different ways. Hacker that use their knowledge within the law of ethical hacking as described in \prettyref{sec:ethical} are named White-Hats. They often help to make systems more secure by cooperating with the responsible persons. A hacker is called a Grey-Hat, if it could happen that laws are breached but nevertheless a serious security problem is the further aim of the operation. By reason of this, it is not possible to classify Grey-Hats into good or bad hackers. In contrast to this, Black-Hats use consciously criminal ways for example to damage a system or to steal personal data.

The idea of this work, is to show the security of encrypted containers by figuring out the passwords. Trying different approaches and possibilities to break into a container are tested and compared. How safe can data be in an encrypted container? Is it possible for everybody to obtain information from inside?

At first, the report gives an introduction about the importance of encryption and the statement of ethical hacking in the society. Afterwards, in \prettyref{ch:relWork} an historic overview is given from the first hacking activities till now. Also related work about the topic is presented. \prettyref{ch:fundamentals} explains basics about en- and decryption that need to be known for the further comprehension. The tested methods and programs are introduced in \prettyref{ch:methods}. \prettyref{ch:eval} shows the results in computation effort that could be achieved. Some limitations appear from these test and are described in \prettyref{ch:limitations}. In the end, \prettyref{ch:conclusion} resumes the gained information.

\section{Motivation}
\label{sec:motivation}

Encryption programs often have and need a reputation of high-security standards. Nevertheless, security gaps are released frequently. Users trust their favourite programs and methods in order to gain a safe feeling for sensitive data. Companies also want to prevent loosing data which could also have a great impact in financial cases. Only fixed persons should have the allowance to read and write specific data. 

Computing machines developed a lot in the past years. With this technical development, machines get faster and more reasonable machines are achievable in general. By reason of this, also the encryption algorithms need to be enhanced. Therefore, different algorithms and programs are tested in the following. 


\section{Ethical Hacking}
\label{sec:ethical}

This project is executed under ethical hacking rules. No ethical hacking laws are breached. The containers are build by the authors with minor unimportant data. By choosing varying encryption algorithms, hash functions and passwords, different possibilities are tested in order to figure out the security status of those. The aim is definitely not to open container in order to find private data but to create a sensibility to password security in containers and to show the possibilities nowadays. 
